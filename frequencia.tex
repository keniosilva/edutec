% Setting up document class and basic packages
\documentclass[a4paper,12pt]{article}
\usepackage[utf8]{inputenc}
\usepackage[T1]{fontenc}
\usepackage{geometry}
\geometry{margin=1in}
\usepackage{booktabs}
\usepackage{parskip}
\usepackage{amsmath}
\usepackage{amsfonts}
\usepackage{graphicx}
\usepackage{hyperref}
\hypersetup{
    colorlinks=true,
    linkcolor=blue,
    filecolor=magenta,
    urlcolor=cyan,
}
\usepackage{noto}

% Configuring document title and headers
\title{Documento de Frequência de Formação}
\author{Unidade Escolar}
\date{\today}

\begin{document}

% Creating title page
\maketitle

% Adding introduction section
\section{Detalhes da Formação}
\textbf{Nome da Formação:} Nome da Formação \\
\textbf{Unidade Escolar:} Nome da Unidade Escolar \\
\textbf{Data de Início:} Data de Início \\
\textbf{Data de Fim:} Data de Fim \\
\textbf{Descrição:} Descrição detalhada da formação.

% Adding frequency table
\section{Registro de Frequência}
\begin{table}[h]
\centering
\begin{tabular}{llc}
\toprule
\textbf{Nome do Participante} & \textbf{Cargo/Função} & \textbf{Presença} \\
\midrule
Participante 1 & Cargo 1 & $\square$ Presente \\
Participante 2 & Cargo 2 & $\square$ Presente \\
Participante 3 & Cargo 3 & $\square$ Presente \\
\bottomrule
\end{tabular}
\caption{Lista de Frequência}
\end{table}

% Adding observations section
\section{Observações}
Espaço para observações adicionais sobre a formação, como comentários sobre a participação, dificuldades encontradas, ou outras informações relevantes.

% Adding signature section
\section{Assinaturas}
\begin{itemize}
    \item \textbf{Coordenador(a):} ______________________________ \\
    \item \textbf{Responsável pela Formação:} ______________________________ \\
\end{itemize}

\end{document}